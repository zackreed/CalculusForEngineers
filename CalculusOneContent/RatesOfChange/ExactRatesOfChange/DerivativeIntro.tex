\documentclass{ximera}

\title{Introduction: Linear Approximation}
\author{Zack Reed}

\begin{document}
\begin{abstract}
    An introduction to one main idea from calculus: linear approximation.
\end{abstract}
\maketitle

\section{Local Linearity: Functions that behave like lines in small enough windows}

One of the central ideas of calculus is the following: many useful functions can be treated as if they are lines when we zoom in close enough. This is called \textit{local linearity}.

In the following GeoGebra applet you seen an example of this intuition in action. The applet shows the graph of the funciton $y=\sin(\theta)$ between $-\pi$ and $\pi$. You can change the value of $\theta$ using the slider in the top left corner, and by default you will see the point on the graph at that value of $\theta$.

If you click the checkbox labeled "Show Local Linearity", you will see a small window around the point $(\theta, y)$ and a box in the top right corner that shows what the function looks like zoomed within that window. Displayed is a line, the point $(\theta, y)$, and two arrows (vectors) labeled $d\theta$ and $dy$. 

% \begin{center}
%     \geogebra{localLinearity}
% \end{center}

The arrows represent the directions and amounts of change to the variables $\theta$ and $y$ as we move along the graph. As we'll see later, $d\theta$ and $dy$ have a very special relationship which makes the graph in the small window \emph{linear} (i.e., like a line). This relationship is that there is some constant $r$ that you multiply by $d\theta$ to get $dy$. We call this a \emph{multiplicative} relationship becase $dy$ and $d\theta$ are related by multiplication. We use the equation 

$$dy = r \cdot d\theta$$

to describe this relationship. We'll call the line that appears in the small window a \emph{linear approximation} of $\sin(\theta)$ at the point $(\theta,y)$, and denote it by $L$. 

$r$ is called the \textit{rate of change} of the linear function. You can see the values for $r$ in the top right corner of the applet. 

Using the applet, answer the following questions about the function $y=\sin(\theta)$:

\begin{enumerate}
    \item What is the value of $y$ when $\theta=0$? 
    \begin{multipleChoice}
        \choice[correct]{0}
        \choice{1}
        \choice{$\pi$}
        \choice{-1}
    \end{multipleChoice}
    \item Select all that are true about the linear approximation $L$ of the function $y=\sin(\theta)$ at the point $(0,0)$.
    \begin{selectAll}
        \choice[correct]{The rate of change of $L$ is $1$.}
        \choice[correct]{The rate $\frac{dy}{d\theta}$ is $1$.}
        \choice{The rate $r=-1$.}
        \choice[correct]{$dy=1\cdot d\theta$.}
        \choice[correct]{$L$ is a linear function.}
        \choice{$L$ is not a linear function.}
        \choice{$L$ at $(0,0)$ is the same as $L$ at $(\pi/2,1)$.}
        \choice{$L$ at $(0,0)$ is the same as $L$ at $(\pi/4,\sqrt{2}/2)$.}
        \choice[correct]{$L$ at $(0,0)$ is the same as $L$ at $(\pi,0)$.}
        \choice{$\frac{dy}{d\theta}$ at $(0,0)$ is the same as $\frac{dy}{d\theta}$ at $(\pi/2,1)$.}
    \end{selectAll}
    \item Which of hte following are true about the function $y=\sin(\theta)$?
\end{enumerate}


\begin{exercise}
    $2+3=\answer{5}$
\end{exercise}

\begin{center} %% Image is found in xmPictures
\includegraphics{missionPatch.jpg}
\end{center}



\end{document}