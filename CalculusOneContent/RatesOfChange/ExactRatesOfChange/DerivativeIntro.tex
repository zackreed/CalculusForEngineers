\documentclass{ximera}

\title{Introduction: Linear Approximation}
\author{Zack Reed}

\begin{document}
\begin{abstract}
    An introduction to one main idea from calculus: linear approximation.
\end{abstract}
\maketitle

\section{Local Linearity: Functions that behave like lines in small enough windows}

One of the central ideas of calculus is the following: many useful functions can be treated as if they are lines when we zoom in close enough. This is called \textit{local linearity}.

In the following GeoGebra applet you seen an example of this intuition in action. The applet shows the graph of the funciton $y=\sin(\theta)$ between $-\pi$ and $\pi$. You can change the value of $\theta$ using the slider in the top left corner, and by default you will see the point on the graph at that value of $\theta$.

If you click the checkbox labeled "Show Local Linearity", you will see a small window around the point $(\theta, y)$ and a box in the top right corner that shows what the function looks like zoomed within that window. Displayed is a line, the point $(\theta, y)$, and two arrows (vectors) labeled $d\theta$ and $dy$. 

% \begin{center}
%     \geogebra{localLinearity}
% \end{center}




\begin{exercise}
    $2+3=\answer{5}$
\end{exercise}

\begin{center} %% Image is found in xmPictures
\includegraphics{missionPatch.jpg}
\end{center}



\end{document}